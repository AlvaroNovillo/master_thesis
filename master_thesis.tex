\documentclass[twoside,nohyper]{tufte-book}

% ams
\usepackage{amssymb,amsmath}

\usepackage{ifxetex,ifluatex}
\usepackage{fixltx2e} % provides \textsubscript
\ifnum 0\ifxetex 1\fi\ifluatex 1\fi=0 % if pdftex
  \usepackage[T1]{fontenc}
  \usepackage[utf8]{inputenc}
\else % if luatex or xelatex
  \makeatletter
  \@ifpackageloaded{fontspec}{}{\usepackage{fontspec}}
  \makeatother
  \defaultfontfeatures{Ligatures=TeX,Scale=MatchLowercase}
  \makeatletter
  \@ifpackageloaded{soul}{
     \renewcommand\allcapsspacing[1]{{\addfontfeature{LetterSpace=15}#1}}
     \renewcommand\smallcapsspacing[1]{{\addfontfeature{LetterSpace=10}#1}}
   }{}
  \makeatother

\fi

% graphix
\usepackage{graphicx}
\setkeys{Gin}{width=\linewidth,totalheight=\textheight,keepaspectratio}

% booktabs
\usepackage{booktabs}

% url
\usepackage{url}

% hyperref
\usepackage{hyperref}

% units.
\usepackage{units}


\setcounter{secnumdepth}{2}

% citations
\usepackage{natbib}
\bibliographystyle{abbrvnat}


% pandoc syntax highlighting

% table with pandoc
\usepackage{longtable,booktabs,array}
\usepackage{calc} % for calculating minipage widths
% Correct order of tables after \paragraph or \subparagraph
\usepackage{etoolbox}
\makeatletter
\patchcmd\longtable{\par}{\if@noskipsec\mbox{}\fi\par}{}{}
\makeatother
% Allow footnotes in longtable head/foot
\IfFileExists{footnotehyper.sty}{\usepackage{footnotehyper}}{\usepackage{footnote}}
\makesavenoteenv{longtable}

% multiplecol
\usepackage{multicol}

% strikeout
\usepackage[normalem]{ulem}

% morefloats
\usepackage{morefloats}


% tightlist macro required by pandoc >= 1.14
\providecommand{\tightlist}{%
  \setlength{\itemsep}{0pt}\setlength{\parskip}{0pt}}

% title / author / date
\title{Data Analytics in Football: Pitch Control and Beyond}
\author{Álvaro Novillo Correas}
\date{2024-02-03}

%----------
%   IMPORTANTE
%----------

% Si nunca has utilizado LaTeX es conveniente que aprendas una serie de conceptos básicos antes de utilizar esta plantilla. Te aconsejamos que leas previamente algún tutorial (puedes encontar muchos en Internet).

% Esta plantilla está basada en las recomendaciones de la guía "Trabajo fin de Máster: Escribir el TFM", que encontrarás en http://uc3m.libguides.com/TFM/escribir
% contiene recomendaciones de la Biblioteca basadas principalmente en estilos APA e IEEE, pero debes seguir siempre las orientaciones de tu Tutor de TFM y la normativa de TFM para tu titulación.

% Encontrarás un ejemplo de TFM realizado con esta misma plantilla en la carpeta "_ejemplo_TFM_2019". Consúltalo porque contiene ejemplos útiles para incorporar tablas, figuras, listados de código, bibliografía, etc.


%----------
%	CONFIGURACIÓN DEL DOCUMENTO
%----------

% Definimos las características del documento y añadimos una serie de paquetes (\usepackage{package}) que agregan funcionalidades a LaTeX.

\documentclass[12pt]{report} %fuente a 12pt

% MÁRGENES: 2,5 cm sup. e inf.; 3 cm izdo. y dcho.
\usepackage[
a4paper,
vmargin=2.5cm,
hmargin=3cm
]{geometry}

% INTERLINEADO: Estrecho (6 ptos./interlineado 1,15) o Moderado (6 ptos./interlineado 1,5)
\renewcommand{\baselinestretch}{1.15}
\parskip=6pt

% DEFINICIÓN DE COLORES para portada y listados de código
\usepackage[table]{xcolor}
\definecolor{azulUC3M}{RGB}{0,0,102}
\definecolor{gray97}{gray}{.97}
\definecolor{gray75}{gray}{.75}
\definecolor{gray45}{gray}{.45}

% Soporte para GENERAR PDF/A --es importante de cara a su inclusión en e-Archivo porque es el formato óptimo de preservación y a la generación de metadatos, tal y como se describe en http://uc3m.libguides.com/ld.php?content_id=31389625. En la carpeta incluímos el archivo plantilla_tfg_2017.xmpdata en el que puedes incluir los metadatos que se incorporarán al archivo PDF cuando lo compiles. Ese archivo debe llamarse igual que tu archivo .tex. Puedes ver un ejemplo en esta misma carpeta.
\usepackage[a-1b]{pdfx}

% ENLACES
\usepackage{hyperref}
\hypersetup{colorlinks=true,
	linkcolor=black, % enlaces a partes del documento (p.e. índice) en color negro
	urlcolor=blue} % enlaces a recursos fuera del documento en azul

% EXPRESIONES MATEMATICAS
\usepackage{amsmath,amssymb,amsfonts,amsthm}

\usepackage{txfonts} 
\usepackage[T1]{fontenc}
\usepackage[utf8]{inputenc}

\usepackage[english]{babel} 
\usepackage[babel, english=american]{csquotes}
\AtBeginEnvironment{quote}{\small}

% diseño de PIE DE PÁGINA
\usepackage{fancyhdr}
\pagestyle{fancy}
\fancyhf{}
\renewcommand{\headrulewidth}{0pt}
\rfoot{\thepage}
\fancypagestyle{plain}{\pagestyle{fancy}}

% DISEÑO DE LOS TÍTULOS de las partes del trabajo (capítulos y epígrafes o subcapítulos)
\usepackage{titlesec}
\usepackage{titletoc}
\titleformat{\chapter}[block]
{\large\bfseries\filcenter}
{\thechapter.}
{5pt}
{\MakeUppercase}
{}
\titlespacing{\chapter}{0pt}{0pt}{*3}
\titlecontents{chapter}
[0pt]                                               
{}
{\contentsmargin{0pt}\thecontentslabel.\enspace\uppercase}
{\contentsmargin{0pt}\uppercase}                        
{\titlerule*[.7pc]{.}\contentspage}                 

\titleformat{\section}
{\bfseries}
{\thesection.}
{5pt}
{}
\titlecontents{section}
[5pt]                                               
{}
{\contentsmargin{0pt}\thecontentslabel.\enspace}
{\contentsmargin{0pt}}
{\titlerule*[.7pc]{.}\contentspage}

\titleformat{\subsection}
{\normalsize\bfseries}
{\thesubsection.}
{5pt}
{}
\titlecontents{subsection}
[10pt]                                               
{}
{\contentsmargin{0pt}                          
	\thecontentslabel.\enspace}
{\contentsmargin{0pt}}                        
{\titlerule*[.7pc]{.}\contentspage}  


% DISEÑO DE TABLAS. Puedes elegir entre el estilo para ingeniería o para ciencias sociales y humanidades. Por defecto, está activado el estilo de ingeniería. Si deseas utilizar el otro, comenta las líneas del diseño de ingeniería y descomenta las del diseño de ciencias sociales y humanidades
\usepackage{multirow} %permite combinar celdas 
\usepackage{caption} %para personalizar el título de tablas y figuras
\usepackage{floatrow} %utilizamos este paquete y sus macros \ttabbox y \ffigbox para alinear los nombres de tablas y figuras de acuerdo con el estilo definido. Para su uso ver archivo de ejemplo 
\usepackage{array} % con este paquete podemos definir en la siguiente línea un nuevo tipo de columna para tablas: ancho personalizado y contenido centrado
\newcolumntype{P}[1]{>{\centering\arraybackslash}p{\verb|#|1}}
\DeclareCaptionFormat{upper}{\verb|#|1\verb|#|2\uppercase{\verb|#|3}\par}

% Diseño de tabla para ingeniería
\captionsetup[table]{
	format=upper,
	justification=centering,
	labelsep=period,
	width=.75\linewidth,
	labelfont=small,
	font=small,
}

%Diseño de tabla para ciencias sociales y humanidades
%\captionsetup[table]{
%	justification=raggedright,
%	labelsep=period,
%	labelfont=small,
%	singlelinecheck=false,
%	font={small,bf}
%}


% DISEÑO DE FIGURAS. Puedes elegir entre el estilo para ingeniería o para ciencias sociales y humanidades. Por defecto, está activado el estilo de ingeniería. Si deseas utilizar el otro, comenta las líneas del diseño de ingeniería y descomenta las del diseño de ciencias sociales y humanidades
\usepackage{graphicx}
\graphicspath{{imagenes/}} %ruta a la carpeta de imágenes

% Diseño de figuras para ingeniería
\captionsetup[figure]{
	format=hang,
	name=Fig.,
	singlelinecheck=off,
	labelsep=period,
	labelfont=small,
	font=small		
}

% Diseño de figuras para ciencias sociales y humanidades
%\captionsetup[figure]{
%	format=hang,
%	name=Figure,
%	singlelinecheck=off,
%	labelsep=period,
%	labelfont=small,
%	font=small		
%}


% NOTAS A PIE DE PÁGINA
\usepackage{chngcntr} %para numeración contínua de las notas al pie
\counterwithout{footnote}{chapter}

% LISTADOS DE CÓDIGO
% soporte y estilo para listados de código. Más información en https://es.wikibooks.org/wiki/Manual_de_LaTeX/Listados_de_código/Listados_con_listings
\usepackage{listings}

% definimos un estilo de listings
\lstdefinestyle{estilo}{ frame=Ltb,
	framerule=0pt,
	aboveskip=0.5cm,
	framextopmargin=3pt,
	framexbottommargin=3pt,
	framexleftmargin=0.4cm,
	framesep=0pt,
	rulesep=.4pt,
	backgroundcolor=\color{gray97},
	rulesepcolor=\color{black},
	%
	basicstyle=\ttfamily\footnotesize,
	keywordstyle=\bfseries,
	stringstyle=\ttfamily,
	showstringspaces = false,
	commentstyle=\color{gray45},     
	%
	numbers=left,
	numbersep=15pt,
	numberstyle=\tiny,
	numberfirstline = false,
	breaklines=true,
	xleftmargin=\parindent
}

\captionsetup[lstlisting]{font=small, labelsep=period}
% fijamos el estilo a utilizar 
\lstset{style=estilo}
\renewcommand{\lstlistingname}{\uppercase{Código}}


%BIBLIOGRAFÍA - PUEDES ELEGIR ENTRE ESTILO IEEE O APA. POR DEFECTO ESTÁ CONFIGURADO IEEE. SI DESEAS USAR APA, COMENTA LAS LÍNEA DE IEEE Y DESCOMENTA LAS DE APA. Si haces cambios en la configuración de la bibliografía y no obtienes los resultados esperados, es recomendable limpiar los archivos auxiliares y volver a compilar en este orden: COMPILAR-BIBLIOGRAFIA-COMPILAR
% Tienes más información sobre cómo generar bibliografía en http://tex.stackexchange.com/questions/154751/biblatex-with-biber-configuring-my-editor-to-avoid-undefined-citations , https://es.sharelatex.com/learn/Bibliography_management_in_LaTeX y en http://www.ctan.org/tex-archive/macros/latex/exptl/biblatex-contrib
% También te recomendamos consultar la guía temática de la Biblioteca sobre citas bibliográficas: http://uc3m.libguides.com/guias_tematicas/citas_bibliograficas/inicio

% CONFIGURACIÓN PARA LA BIBLIOGRAFÍA IEEE
\usepackage[backend=biber, style=ieee, isbn=false,sortcites, maxbibnames=5, minbibnames=1]{biblatex} % Configuración para el estilo de citas de IEEE, recomendado para el área de ingeniería. "maxbibnames" indica que a partir de 5 autores trunque la lista el primero (minbibnames) y añada "et al." tal y como se utiliza en el estilo IEEE.

%CONFIGURACIÓN PARA LA BIBLIOGRAFÍA APA
%\usepackage[style=apa, backend=biber, natbib=true, hyperref=true, uniquelist=false, sortcites]{biblatex}
%\DeclareLanguageMapping{spanish}{spanish-apa}

\addbibresource{bibliografia.bib} % llama al archivo bibliografia.bib que utilizamos de ejemplo

\begin{document}

\maketitle

\begin{abstract}
\noindent \textbackslash colchunk\{

\newthought{Mobile user devices'} TODOTODO
\end{abstract}


{
\setcounter{tocdepth}{1}
\tableofcontents
}

\listoffigures

\hypertarget{dedication}{%
\chapter{Dedication}\label{dedication}}

\hypertarget{the-pool-of-tears}{%
\chapter{The pool of tears}\label{the-pool-of-tears}}

\hypertarget{a-caucus-race-and-a-long-tale}{%
\chapter{A caucus-race and a long tale}\label{a-caucus-race-and-a-long-tale}}

\listoffigures

\hypertarget{dedication-1}{%
\chapter{Dedication}\label{dedication-1}}

\hypertarget{the-pool-of-tears-1}{%
\chapter{The pool of tears}\label{the-pool-of-tears-1}}

\hypertarget{a-caucus-race-and-a-long-tale-1}{%
\chapter{A caucus-race and a long tale}\label{a-caucus-race-and-a-long-tale-1}}

Master Degree in...\\
Academic Year (e.g.~2018-2019)\\
\emph{Master Thesis}

Data Analytics in Football: Pitch Control and Beyond\\

\begin{center}\rule{0.5\linewidth}{0.5pt}\end{center}

\hfill\break
Author's complete name\\

1st Tutor complete name\\
2nd Tutor complete name\\
Place and date\\

\hfill\break
\includegraphics[width=4.2cm,height=\textheight]{imagenes/creativecommons.png}\\
\emph{\[Include this code in case you want your Master Thesis published in
Open Access University Repository\]}\\
This work is licensed under Creative Commons \textbf{Attribution -- Non
Commercial -- Non Derivatives}

\hypertarget{dedication}{%
\chapter*{Dedication}\label{dedication}}
\addcontentsline{toc}{chapter}{Dedication}

\hypertarget{introduction}{%
\chapter{Introduction}\label{introduction}}

\hypertarget{data-analytics-in-football}{%
\chapter{Data Analytics in football}\label{data-analytics-in-football}}

The digital revolution is currently one of the most significant
challenges of our time, altering numerous aspects of society. Football,
in particular, has also been influenced by this transformation.
Technological advancements and digitalization have resulted in a swift
upsurge in the number of measuring devices, data collection and volumes
of data. The leading data companies worldwide, including IBM, Intel, SAP
and Microsoft, are vying for superior data analytics tools and
leveraging sports as an example domain to showcase their products and
brand power \citep{1}.

The practice of data analytics in football has a long history, dating
back to the post-World War II era, when data collection and analysis was
undertaken manually using pencil and paper \citep{1}. It wasn't until
Moneyball was published in 2003 that significant progress began to
emerge: The book, "The Art of Winning an Unfair Game" introduced
sports analytics to a broader audience. It illustrated the use of data
analytics in identifying undervalued players and constructing a
successful team. Since then, data analytics has become an integral
component of sport, football inclusive \citep{1}.

One of the best examples of data analytics being applied to sports is
basketball. Teams use data to analyze player performance, identify
strengths and weaknesses, and develop strategies to win games \citep{2}. They
use in-memory analytics, visualization, the cloud, mobility, camera
footage, and sensors to transform their game. This performance analyses
are of vital importance to a team, aiming to reduce expenditure, enhance
team worth and refine processes across all levels and segments of
operations. The German Football Association (DFB) and the National
Basketball Association (NBA) are two unique cases of digital
transformation from the sports world. Successful teams turn player
performance data into action and gain a competitive advantage.

Over the last years, football analytics has gained significant
popularity, aiming to delve deeper into the game by utilizing advanced
data analysis techniques to optimize team and player performance. This
chapter examines the various areas of football where data can be used
for analysis, alongside the commonly found data types within this
industry.

\hypertarget{statistical-analysis-in-football}{%
\section{Statistical analysis in football}\label{statistical-analysis-in-football}}

When discussing sports analytics in football, the first metric that
often springs to mind is the Expected Goals (xG) ratio. This statistical
indicator is a predictive Machine Learning (ML) model used to assess the
likelihood of scoring for every shot made in the game. In the context of
each shot, the xG model computes the scoring probability, leveraging a
set of event parameters.

Wyscout xG model, for example, encompass the shot's spatial coordinates,
the assisting player's position, the striking player's use of foot or
head, the type of assist involved, the occurrence of a dribble by either
a field player or the goalkeeper immediately preceding the shot, whether
the shot arises from a set piece, whether it transpires during a
counterattack or in a transitional phase of play, and the subjective
assessment of shot danger as determined by a designated tagger. The
amalgamation of these parameters serves as the foundation for training
the xG model using historical Wyscout data, culminating in the
prediction of the likelihood of a given shot resulting in a goal
\citep{wyscout}.

The probabilities range from 0 to 1. Thus, a shot with an xG value of
0.1 has a 10\% chance of being scored. Penalties have a fixed xG value of
0.76.

Fig. \protect\hyperlink{f2.1}{2.1} provides a
visual representation of the cumulative development of expected goals
(xG) during the Eibar - Malaga match, which took place on January 15th,
2023 in Spain's second division. Each data point on the graph
corresponds to a shot made by both teams over the course of the game,
offering a comprehensive overview of the evolving scoring opportunities
and outcomes throughout the duration of the game.

We can also make a shot map of each shot, to illustrates the spatial
distribution of shot locations taken by both teams during the game. Fig.
\protect\hyperlink{f2.2}{2.2}. The size of each
data point corresponds to the expected goals (xG) generated for the
respective shots, providing insights into the perceived scoring
potential. Goals scored are visually highlighted with straight lines,
indicating the trajectory the ball followed as it found its way into the
opponent's net.

Cumulative development of expected goals (xG) during the
Eibar-Malaga match, held on January 15th in Spain's second division.
Each point denotes a shot made by both teams throughout the game.
Vertical dashed lines indicate the goal scored, displaying the player
and the corresponding score at that specific moment of the match.

Shot map of the Eibar (blue, left) - Malaga (red, right)
football match. The locations of the points indicate where shots were
taken. The size of each point is proportional to the expected goals (xG)
generated. Shots that resulted in goals are depicted with a straight
line, representing the path the ball took to enter the opponent's
net.

Analyses such as the one above are carried out using the most common
source of data in football: \textbf{Events} datasets.

\hypertarget{events-data}{%
\subsection{Events data}\label{events-data}}

Event data describes specific, human-defined events during a match,
including passes, shots, and fouls. It is captured by human annotators
from various providers. However, this manual process is time-consuming
and typically requires three individuals:

The data collection process is carried out by professional video
analysts (known as operators), who are specialists in football data
collection, using proprietary software (the tagger). The tagger has
undergone several years of development and improvement and is regularly
updated to ensure the highest level of performance is achieved. To
ensure accurate data collection when tagging events in soccer matches,
three operators are assigned: one per team and one supervising the
output of the entire match. This process is based on analysis of the
tagger and soccer match videos. When near-live data delivery is
necessary, a team of four operators may be utilized, with one operator
dedicated to hastening the collection of complex events that require
additional, specific attributes or a quick review \citep{3}.

This type of data structure can be used in a number of ways: it can be
used to measure team performance through general statistics extracted
from event datasets, such as goals, fouls, xG, etc. It can also be used
to create advanced analysis of the team using ensembles of mathematical
tools.

The analysis of the match is furthered through the use of graph theory,
Buldú et,al \citep{Buldu}, A. Novillo et. al \citep{NOVILLO2024114355}. Combining
different elements of the events dataset, we can create a graph
corresponding to the passing network of each team, allowing us to
understand the passing structure of both teams.

Figs. \protect\hyperlink{f2.3}{2.3} and
\protect\hyperlink{f2.4}{2.4} illustrate the
passing networks observed in the Eibar versus Málaga football match,
providing insight into the passing interactions and tactical strategies
used by both teams. The nodes in the graphs represent individual players
who participated in the match for each team. The nodes are sized
according to their degree, which represents the amount of ingoing and
outgoing passes. The node position corresponds to the average passing
position of each player. Substitutes are represented by yellow nodes,
and links are created if there have been at least 5 passes made in that
direction between two players. The edge's width corresponds to the
amount of passes made in that direction between the two players.

Representation of the Eibar passing networks of the match
Eibar - Málaga. Nodes represent players, edges represent passes between
players. The position of the players in the field is their average
passing position. The size of the nodes reflects the number of ingoing
and outgoing passes (i.e.~node's degree), while the size of the edges is
proportional to the number of passes between the players. Substitutes
are represented in yellow. A connection is set if those players share at
least 5 passes. The edge's width is proportional to the amount of passes
made in that direction between the two players.

Representation of the Málaga passing networks of the match
Eibar - Málaga. Nodes represent players, edges represent passes between
players. The position of the players in the field is their average
passing position. The size of the nodes reflects the number of ingoing
and outgoing passes (i.e.~node's degree), while the size of the edges is
proportional to the number of passes between the players. Substitutes
are represented in yellow. A connection is set if those players share at
least 5 passes. The edge's width is proportional to the amount of passes
made in that direction between the two players.

Analysis as the former can be conducted \emph{in real-time}\footnote{Opta uses a combination of human annotation, computer vision, and
  AI modelling to offer real-time data at various levels of detail
  based on customer requirements. In our situation, the data feed
  updates itself when an event such as a goal, foul or pass occurs;
  otherwise, it updates every 90 seconds. \citep{opta}} during the
match using appropriate data sources. Additionally, we could examine
Eibar's macro situation during the 2022-2023 season to better comprehend
how this micro-statistics contribute to the overall perception of the
team.

Fig. \protect\hyperlink{f2.5}{2.5} presents the
expected goals (xG) produced by Eibar in all matches played against
their opponents. It is noticeable that Eibar has generated a higher xG
when playing at their Home stadium, on average. In Fig.
\protect\hyperlink{f2.6}{2.6} an overview of
Eibar's performance against other teams in the Second Division is
presented. It can be observed that Eibar ranks third in generating xGs
against their opponents.

Expected Goals (xG) and Expected Goals Against (xGA) per
match. Codes: Home Matches (Diamonds), Away Matches (Circles), Wins
(Green), Draws (Blue), Losses (Red). Matches above the dashed lines
represent those matches where Eibar has generated more xG than the
opponent.

Ranking of the average differences in Expected Goals Scored
(xG) Minus Expected Goals Conceded by Opponents (xGa) per
team.

We've just discussed some of the many statistics that can be inferred
from this data sources to characterize the team performance, such as
possession, pressure, duels, fouls, etc. Due to limitations in space and
scope, however, we are unable to provide a more in-depth analysis of
these measures\footnote{For additional information regarding this type of analysis, we
  have included two reports outlining Eibar's performance during the
  2022/2023 season in the annex. In these reports, an array of metrics
  is presented to appraise team performance relative to other teams in
  the league, as well as each individual player's performance.}.

Although event datasets supply beneficial information regarding the
team's overall performance, deeper scrutiny can be conducted via
tracking data, which consists of the players' and ball's position and
movement during the match. Tracking data can offer additional insight
into both the physical and tactical aspects of the game.

\hypertarget{tracking}{%
\subsection{Tracking}\label{tracking}}

Tracking data offers a more comprehensive perspective than event data by
providing access to information on all players, their trajectories, and
velocities. This allows for the analysis of off-ball players and team
dynamics, resulting in a more nuanced understanding of the game.

There are two main techniques for obtaining tracking data, which decide
its classification: Image detection algorithms extract players'
positions from the match broadcast and infer locations of concealed
players, whereas optical tracking employs a specialized camera system
installed on the field to record players' data. Our research will
concentrate on the latter method, as it offers more precise and
statistically informative data.

Mediacoach® utilises the Tracab Optical Tracking system to obtain
on-the-pitch player positions. This multi-camera system captures each
player's position at 25 frames per second. The system consists of three
units, each with a resolution of 1920x1080 pixels, producing a panoramic
picture that generates a stereoscopic view for triangulating the players
and ball. In case of a temporary loss of any location, a skilled
operator adjusts the players' positions. The datasets obtained by the
Mediacoach® system have been validated in advance using GPS
\citep{Felipe2019ValidationOA}.

As a demonstration of the amount of information this kind of data can
provide us, we have painted a frame from Atlético de Madrid (Blue) -
Getafe (Red) game from the Spanish 2019 League. Fig.
\protect\hyperlink{f2.7}{2.7}. The ball is shown
as a black dot. Referees are shown as yellow squares. Purple arrows
represent the speed vectors of the players.

As a first approach, we can see where the players have moved during the
game by plotting the heat map of the ball over the whole game. Note that
we always keep the direction of play from left to right, so the home
team will always be placed on the left side of the field and the away
team on the right. Fig. \protect\hyperlink{f2.8}{2.8}.

A frame of tracking data from a football match. The home
team is shown in blue, the away team in red. The ball is shown as a
black dot. Referees are shown as yellow squares. Purple arrows represent
the speed vectors of the players.

Heatmap of the ball position during the Atlético de Madrid -
Getafe game under study. Note that we always keep the direction of play
from left to right, so the home team will always be placed on the left
side of the field and the away team on the right.

Heatmaps such as the one shown in Fig. \protect\hyperlink{f2.8}{2.8} have proved useful to a wide range of stakeholders,
including technical staff, journalists and passionate fans, who have all
sought to gain insight into the positioning of players during matches.
Such granular information about a player's spatial occupation of the
pitch serves as a valuable resource for technical staff, both in match
preparation by analysing the expected positions of opponents, and in
post-match evaluation to assess the extent to which the tactical game
plan has been adhered to. \citep{Garrido_2022}.

This data-driven approach enhances the understanding of the sport and
its strategic nuances, fostering a deeper appreciation of the game's
intricacies. For this reason, tracking data has gained popularity as a
valuable tool in football analysis, providing insights into player
performance and team strategy. These advanced metrics can be applied to
specific games, as in the analysis presented earlier (Fig.
\protect\hyperlink{f2.8}{2.8}), or to an ensemble
of them to provide a comprehensive view of general player behavior under
different parameters. In the following sections, we present a selection
of these metrics, both physical and tactical, to provide context for our
proposed Offside Control metric across multiple games.

\hypertarget{physical-metrics}{%
\subsection{Physical metrics}\label{physical-metrics}}

The analysis of player speed in football is a topic of great interest in
sports science and performance analysis. Understanding the role and
development of sprinting speed in football is crucial to optimizing
player performance. Using tracking data, we can make general statistics
about the patterns that are evident in the speed of football players. In
Fig. \protect\hyperlink{f2.9}{2.9} we analyzed 362
players in 100 matches with a total of 1,502,560 frames. To ensure the
interpretability of the results, we only included player data for those
matches in which that particular player played the entire match. The
resulting speed data showed a bimodal distribution that can be
identified with the two main actions performed by the players during the
game: walking and jogging.

Probability distribution function of the velocities for
(A) all different field players and
(B) referees during a match. Values are averaged over
{100} matches

Having characterized the general distributions underlying on players'
movement, we can delve into more particular investigations that offer
significant viewpoints on game strategy and player effectiveness.

One such question may be to check if players' performance is consistent
across the entire game. That is, if there are no significant differences
between the games' first and second halves.

In the hypothesis contrast carried out in Fig.
\protect\hyperlink{f2.10}{2.10}, significant
variations are evident between the average speeds during the yellow
regime (walking) and green regime (jogging) for all positions, between
the first and second halves of the game. This leads to the conclusion
that the average speed is lower in the latter phase. It is noteworthy
that there was a higher occurrence of values within the walking regime
in the second half compared to the first, as evidenced by the
distribution difference.

We can also observe the same outcome by analysing the mean distance
covered by each player in each half, as measured in
\protect\hyperlink{f2.11}{2.11}. The graphs
illustrate that most players remain below the diagonal dashed black
line, suggesting that the distance covered during the first part of the
game is generally grater than the one covered in the second half

Speed distribution difference between the first and second
half of the game. (A) Goalkeeper. (B)
Defender (C) Midfielder. (D) Forward.
Yellow-shadowed areas represent the values taken for the first peak.
Green-shadowed areas for the second. For each position and peak, the
average speed is smaller in the second half, with a {pvaluep \textless{} 0.001}.
* Average speed is greater in the first
half.

Distance covered in the first and second half of the game
by the players (while walking). (A) Goalkeeper.
(B) Defender (C) Midfielder.
(D) Forward. The percentages shown account for the
number of points that lie below/above the diagonal line

\hypertarget{tactical-metrics}{%
\subsection{Tactical metrics}\label{tactical-metrics}}

After analysing the physics of player movement, a sophisticated
framework is needed to decode how the intricate movements of players
translate to the soccer field.

These models provide a scientific perspective for analysing player
positioning, decision-making, and team dynamics, illuminating the
complex interactions that occur during a match.

\textbf{Pitch control} models \citep{Spearman} in soccer are an essential tool for
researchers, coaches, and analysts who aim to understand the game's
nuances at a granular level.

\hypertarget{pitch-control}{%
\subsubsection{Pitch Control}\label{pitch-control}}

The Pitch Control (PC) at a given location represents the probability of
a player or team gaining control of the ball if it moves directly to
that location. PC models simulate the dynamics of the ball and the
players to evaluate which player would control the ball if it moves to
any location on the pitch at any moment. The model captures not only the
players' current positionbut also their movement. When players are
running at high speeds, they are more likely to control the space they
are moving into rather than the space they currently occupy.

To construct this model, we must calculate the following for a given
location on the pitch:

\begin{itemize}
\item
  How long it would take for the ball to reach to the position of
  interest (from its starting position).
\item
  How long would it take for each player to get to that position.
\item
  What is the total \emph{probability} that each team will control the ball
  \emph{after} both the players and the ball have arrived at the desired
  position?
\end{itemize}

The ball is set to move at a constant speed of \(v_b = 54\) \(km/h\).
Therefore, the time taken to arrive at the location of interest can be
easily calculated as \(t_{b,arr} = \Delta x_b/v_b\), where \(\Delta x_b\) is
the distance between the initial and final positions of the ball.

When considering how long it will take the players to reach the target
position, given their initial position and speed, players are assumed to
only have a maximum speed of \(v_{max,p} = 18\) \(km/h\). This upper limit
should not be misunderstood as the maximum speed at which players can
move, but rather as an estimate of the maximum speed at which they are
likely to move when trying to control the ball.

To compute the player's expected arrival time,
\(\tau_{exp}(\vec{r} ; t_r)\), we use a simple approximation consisting of
a two-step process:

\begin{itemize}
\item
  There is an initial \emph{reaction time} of \(t_{r} = 0.7\) seconds. This
  is approximately the time it takes a player moving at maximum speed
  to come to a complete stop. During this reaction time, we assume
  that players continue to move along their current trajectory without
  changing speed or direction (reaching a position \(\vec{r}_{react}\)).
\item
  After this time, we assume that the player runs directly towards the
  ball at his maximum speed of \(v_{max,p}\).
\end{itemize}

\[\tau_{exp}(\vec{r} ; t_r) = t_r + \frac{|\vec{r} - \vec{r}_{react}|}{v_{max,p}}
    \label{exp_arr_time}\]

Once we computed the time it takes for the ball and the players to get
to the target location, we need to look at how long it will take each
player to control the ball. To do so, we will assume that controlling
the ball is a stochastic process that follows an exponential
distribution with a fixed rate \(\lambda\), with units of \(1/s\). Thus, for
any differential time \(\Delta t\) that a player is near the ball, he has
a probability of \(\lambda \cdot \Delta t\) of controlling the ball.

Cumulative distribution function of the time to control the
ball. The parameters choice is based on {}

So far, the model assumes that we know exactly when each player will
arrive at the target location. However, we introduce some uncertainty,
labelled \(\sigma\), in the arrival time of the players. The reason for
including such temporal variability in our model is to account for some
effects that have not been explicitly modelled, such as player effort.
Thus, the probability of a player intercepting the ball at time T is
given by the cumulative distribution function of the sigmoid
distribution:

\[F_{\text {int }}(\vec{r},T;\sigma, t_r)=\frac{1}{1+e^{-\frac{T- \tau_{exp}(\vec{r} ; t_r)}{\sqrt{3} \sigma / \pi}}}
    \label{sigmoid}\]

Cumulative distribution function of the probability to
intercept the ball at a certain time T {[}sigmoid{]}. {}.

Finally, the differential equation used to compute the control
probability for each player at a given location \(r\), at time t is
\citep{Spearman}:
\[\frac{d P P C F_j}{d t}\left(T, \vec{r} , \sigma, \lambda_j, t_r\right)=\left(1-\sum_k P P C F_k\left(t, \vec{r} , \sigma, \lambda_k\right)\right) F_{int,j}(t, \vec{r}  , \sigma, t_r) \lambda_j
\label{PC_eq}\] where \(PPCF_j\) is the Potential Pitch Control Field of
player \(j\). \(F_{int,j}(t, \vec{r}, T ; \sigma, t_r)\) is the probability
that player \(j\) can reach the target location \(r\) in a given time \(t\),
and \(\lambda_j\) is the control rate of such a player. We assign the
goalkeepers to have a higher control rate, \(\lambda_{GK} = 12.9\)
\(s^{-1}\), to ensure that they are likely to claim the ball if it is near
them and also to account for the ability of grabbing the ball with their
hands. Importantly, note that
\(\sum_k P P C F_k\left(T, \vec{r} , \sigma, \lambda_k\right)\) accounts
for the sum of the Potential Pitch Control Field of the rest of the \(k\)
players on the pitch at time \(t\).

By integrating Eq. \protect\hyperlink{PC_eq}{\[PC_eq\]} over \(t \in \left[ t_{ball},t_{ball} + 10 \right]\)s
and taking \(P P C F_j\left(t, \vec{r} , \sigma, \lambda_j\right) = 0\) at
the beginning of the integration, the probability of control per player
is obtained. This probability is then extracted along all the pitch,
obtaining a pitch control surface, which can be plot and interpreted by
scientists, technical staff or players.

\hypertarget{offside-control}{%
\chapter{Offside Control}\label{offside-control}}

This study introduces the Offside Control (OC) parameter, which is
derived from a modification of the Potential Pitch Control Field to
analyse the pitch control generated by teams/players beyond the offside
line. The offside line is identified as the location of the
second-to-last defender, and the attacking players in both valid and
offside positions are detected at every frame. The OC is calculated as
the pitch control generated by the attacking team/player after the
offside line. Additionally, if a player is in an offside position, their
contribution is quantified as Ineffective Offside Control (IOC).
Conversely, when a player is before the offside line and generates pitch
control after it, it is referred to as Effective Offside Control (EOC).

Fig. \protect\hyperlink{f3.1}{3.1} illustrates the
concept of effective and ineffective Offside Control. The players of
both teams are represented in cyan (attacking from left to right) and
red (attacking from right to left), corresponding to the home and away
team, respectively. The regions of the pitch controlled by the home team
are indicated in blue, while the red regions are controlled by the away
team. The offside position is indicated by the vertical dashed line. The
wide grey line surrounds the areas controlled by the attacking team at
the back of the offside line. Player \(11\), is located behind the offside
line, which is an invalid position and generates an offside infraction.
In contrast, player \(18\) is in a correct position, and the offside
control generated behind the offside line is classified as effective
(EOC).

In the study, we computed the Offside Control (both effective and
ineffective) at a rate of 2 frames per second for each of the 100
matches analyzed, resulting in a total of 1,251,934 frames analysed.
Doing so, we managed to characterize the Offside Control of 442 players
in total. To efficiently discretize the space, the computations of the
Offside Control were confined to the attacking half of the pitch of the
team in possession of the ball. This method considerably decreased the
model's computation time while maintaining a high spatial resolution of
50 by 32 divisions of the attacking half of the pitch.

Example of effective (EOC) and ineffective (IOC) Offside
Control (areas surrounded by a wide grey line). The offside line is the
vertical dashed line. Note that player {11} is offside (generating IOC), while player
{18} is not (creating EOC).

To obtain the OC generated by the team, simply sum the individual
contributions of the players during the match. Figure 3.2 shows two
examples of the different spatial distribution of the EOC in two
different matches of the same team (Team 1). In MATCH A, we can see how
the exerted EOC by Team 1 is deep and close to the opponent's area (Team
2), whereas in the second match, MATCH B, the EOC is longitudinal across
the side lanes. The bar chart on the right-hand side of each plot
displays the average EOC generated by each team per unit of effective
time, considering only instances where the ball was in play. In both
cases, Team 1 created more pressure after the offside lane than their
opponents.

Cumulative Effective Offside Control distribution over the
course of two matches. The same team (Team 1) (left) generates offside
control in two different ways at two different matches (A,B). Bars on
right-hand side show the average effective offside control generated by
each team per unit of effective time. {ºbf (cambiar numeros por
letras y viceversa, segun texto principal)}

The correlation between the average OC per unit of effective time and
the final classification following the tenth match of the season is
shown in Fig. \protect\hyperlink{f3.3}{3.3}. The
scatter plot shows both the generated and received average OC per unit
of effective time, with each label and point size representing the
team's classification after the tenth match of the season. Teams above
the diagonal black dashed line have generated more OC than their
opponents.

Average generated Offside Control vs the received one. The
values correspond to the average accumulated OC per effective unit of
time. The labels and the size of each point correspond to their final
classification after the tenth match of the season. Teams above the dark
dashed line have accumulated more offside control than the one generated
by their rivals. The fitted equation is {y =  − 0.86x + 285.42}, with
an {R2 = 0.48} and
{RMSE = 14.96}

The distribution of pressure among teams in the league is heterogeneous,
with some higher-ranking teams exerting more pressure than their
opponents. There are some exceptions, with lower-ranking teams also
exerting more pressure than their opponents. However, it is clear that
OC alone cannot fully explain team rankings. A more comprehensive study
is needed to include additional variables that could better explain the
rankings.

Fig. \protect\hyperlink{f3.4}{3.4} illustrates the
progression of Cumulative Offside Control during a match. The cumulative
sum is calculated using a centered moving window comprising 600 frames.
The blue line represents the total offside control of the home team,
including both effective and ineffective contributions, per unit of
effective time.

The cumulative rate of the OC decreases towards the latter half of the
game, resulting in the home team losing pressure on the opponent and
conceding a couple of goals.

Cumulative Offside Control evolution over the course of a
match. Cumulative evolution of offside control over the course of a
match. The cumulative sum has been taken using a centered moving window
of 600 frames. the total offside control of the home team is represented
in blue (including both effective/ineffective contributions) per unit of
effective time.

By examining the spatial distribution of the total OC generated across
all matches played and studying the forwards' positions relative to the
offside line, we find that OC is contingent on the forwards' position
relative to the offside line, thus proving useful in characterising
different forward playstyles. Figure \protect\hyperlink{f3.5}{3.5} shows that Forward A spends more time preceding the
offside line (negative values) than Forward B. The EOC is normalised by
the time played and encompasses a minimum of 6 matches.

Effective Offside Control of two different forwards A/B. On
the left, positions of the pitch where two different forward players A/B
generate Effective Offside Control (EOC). On the right, a probability
distribution function of the distance to the offside line for the two
different forwards. EOC is normalized by the time played, containing at
least 6 matches. Note how the spatial distribution of the EOC depends on
their position with respect to the offside line. We can see of forward A
spends more time before the offside line (negative values) than forward
B.

\hypertarget{conclusion}{%
\chapter{Conclusion}\label{conclusion}}

Data-driven approaches for analysing football performance using event
and tracking data have gained popularity in the last decade. In this
project, we previously discussed how metrics such as Expected Goals
(xG), derived from events datasets, have impacted our understanding of
the game, becoming an essential tool for evaluating team performance.

However, tracking datasets allow us to develop advanced physical and
tactical metrics that capture the aspects of player and team
performance. In this context, Pitch Control arises as sophisticated
approach to quantifying and analyzing the intricate dynamics of soccer
from a scientific perspective. Pitch control models aim to provide a
quantitative framework for assessing player movements, spatial control,
and decision-making on the field. They utilize positional data of
players and the ball to determine the likelihood of a player gaining
control. At any given moment, the position of the ball can be determined
by taking into account variables such as player speed, direction, and
relative positioning to their opponents and teammates. This modelling
approach provides tactical insights and spatial relationships between
players, the ball, and the field. By studying how players' movements
affect pitch control, teams can optimize their formations, pressing
patterns, and passing strategies.

From these models, metrics such as the one presented in this project,
Offside Control, quantify a player's pitch control contributions locally
(in this case, after the offside line). This makes it possible to
identify strengths and weaknesses in team pressure, track improvements
over time, and make data-driven decisions regarding player selection and
development. Clubs and talent scouts can use these metrics derived from
Pitch Control models to evaluate potential signings. In the case of
Offside Control, these models can aid in identifying talented players by
quantifying their ability to control space and make effective decisions.

Currently, tracking dataset approaches are not measured in real-time,
unlike events dataset. The development of new technologies for football
analytics, such as the recently announced TRACAB Optical system, shows
promise for research in this area \citep{TRACAB}. The system uses advanced AI
to provide live in-game tracking data and unprecedented levels of
precision to identify and track players' body parts. This technology can
power new categories of performance analysis and skeletal modeling for
exciting new applications. As the industry evolves, so will the
sophistication of these models, offering a deeper understanding of the
game and enhancing its scientific appeal.

Master Degree in...\\
Academic Year (e.g.~2018-2019)\\
\emph{Master Thesis}

Data Analytics in Football: Pitch Control and Beyond\\

\begin{center}\rule{0.5\linewidth}{0.5pt}\end{center}

\hfill\break
Author's complete name\\

1st Tutor complete name\\
2nd Tutor complete name\\
Place and date\\

\hfill\break
\includegraphics[width=4.2cm,height=\textheight]{imagenes/creativecommons.png}\\
\emph{\[Include this code in case you want your Master Thesis published in
Open Access University Repository\]}\\
This work is licensed under Creative Commons \textbf{Attribution -- Non
Commercial -- Non Derivatives}

\hypertarget{dedication}{%
\chapter*{Dedication}\label{dedication}}
\addcontentsline{toc}{chapter}{Dedication}

\hypertarget{introduction-1}{%
\chapter{Introduction}\label{introduction-1}}

\hypertarget{data-analytics-in-football-1}{%
\chapter{Data Analytics in football}\label{data-analytics-in-football-1}}

The digital revolution is currently one of the most significant
challenges of our time, altering numerous aspects of society. Football,
in particular, has also been influenced by this transformation.
Technological advancements and digitalization have resulted in a swift
upsurge in the number of measuring devices, data collection and volumes
of data. The leading data companies worldwide, including IBM, Intel, SAP
and Microsoft, are vying for superior data analytics tools and
leveraging sports as an example domain to showcase their products and
brand power \citep{1}.

The practice of data analytics in football has a long history, dating
back to the post-World War II era, when data collection and analysis was
undertaken manually using pencil and paper \citep{1}. It wasn't until
Moneyball was published in 2003 that significant progress began to
emerge: The book, "The Art of Winning an Unfair Game" introduced
sports analytics to a broader audience. It illustrated the use of data
analytics in identifying undervalued players and constructing a
successful team. Since then, data analytics has become an integral
component of sport, football inclusive \citep{1}.

One of the best examples of data analytics being applied to sports is
basketball. Teams use data to analyze player performance, identify
strengths and weaknesses, and develop strategies to win games \citep{2}. They
use in-memory analytics, visualization, the cloud, mobility, camera
footage, and sensors to transform their game. This performance analyses
are of vital importance to a team, aiming to reduce expenditure, enhance
team worth and refine processes across all levels and segments of
operations. The German Football Association (DFB) and the National
Basketball Association (NBA) are two unique cases of digital
transformation from the sports world. Successful teams turn player
performance data into action and gain a competitive advantage.

Over the last years, football analytics has gained significant
popularity, aiming to delve deeper into the game by utilizing advanced
data analysis techniques to optimize team and player performance. This
chapter examines the various areas of football where data can be used
for analysis, alongside the commonly found data types within this
industry.

\hypertarget{statistical-analysis-in-football-1}{%
\section{Statistical analysis in football}\label{statistical-analysis-in-football-1}}

When discussing sports analytics in football, the first metric that
often springs to mind is the Expected Goals (xG) ratio. This statistical
indicator is a predictive Machine Learning (ML) model used to assess the
likelihood of scoring for every shot made in the game. In the context of
each shot, the xG model computes the scoring probability, leveraging a
set of event parameters.

Wyscout xG model, for example, encompass the shot's spatial coordinates,
the assisting player's position, the striking player's use of foot or
head, the type of assist involved, the occurrence of a dribble by either
a field player or the goalkeeper immediately preceding the shot, whether
the shot arises from a set piece, whether it transpires during a
counterattack or in a transitional phase of play, and the subjective
assessment of shot danger as determined by a designated tagger. The
amalgamation of these parameters serves as the foundation for training
the xG model using historical Wyscout data, culminating in the
prediction of the likelihood of a given shot resulting in a goal
\citep{wyscout}.

The probabilities range from 0 to 1. Thus, a shot with an xG value of
0.1 has a 10\% chance of being scored. Penalties have a fixed xG value of
0.76.

Fig. \protect\hyperlink{f2.1}{2.1} provides a
visual representation of the cumulative development of expected goals
(xG) during the Eibar - Malaga match, which took place on January 15th,
2023 in Spain's second division. Each data point on the graph
corresponds to a shot made by both teams over the course of the game,
offering a comprehensive overview of the evolving scoring opportunities
and outcomes throughout the duration of the game.

We can also make a shot map of each shot, to illustrates the spatial
distribution of shot locations taken by both teams during the game. Fig.
\protect\hyperlink{f2.2}{2.2}. The size of each
data point corresponds to the expected goals (xG) generated for the
respective shots, providing insights into the perceived scoring
potential. Goals scored are visually highlighted with straight lines,
indicating the trajectory the ball followed as it found its way into the
opponent's net.

Cumulative development of expected goals (xG) during the
Eibar-Malaga match, held on January 15th in Spain's second division.
Each point denotes a shot made by both teams throughout the game.
Vertical dashed lines indicate the goal scored, displaying the player
and the corresponding score at that specific moment of the match.

Shot map of the Eibar (blue, left) - Malaga (red, right)
football match. The locations of the points indicate where shots were
taken. The size of each point is proportional to the expected goals (xG)
generated. Shots that resulted in goals are depicted with a straight
line, representing the path the ball took to enter the opponent's
net.

Analyses such as the one above are carried out using the most common
source of data in football: \textbf{Events} datasets.

\hypertarget{events-data-1}{%
\subsection{Events data}\label{events-data-1}}

Event data describes specific, human-defined events during a match,
including passes, shots, and fouls. It is captured by human annotators
from various providers. However, this manual process is time-consuming
and typically requires three individuals:

The data collection process is carried out by professional video
analysts (known as operators), who are specialists in football data
collection, using proprietary software (the tagger). The tagger has
undergone several years of development and improvement and is regularly
updated to ensure the highest level of performance is achieved. To
ensure accurate data collection when tagging events in soccer matches,
three operators are assigned: one per team and one supervising the
output of the entire match. This process is based on analysis of the
tagger and soccer match videos. When near-live data delivery is
necessary, a team of four operators may be utilized, with one operator
dedicated to hastening the collection of complex events that require
additional, specific attributes or a quick review \citep{3}.

This type of data structure can be used in a number of ways: it can be
used to measure team performance through general statistics extracted
from event datasets, such as goals, fouls, xG, etc. It can also be used
to create advanced analysis of the team using ensembles of mathematical
tools.

The analysis of the match is furthered through the use of graph theory,
Buldú et,al \citep{Buldu}, A. Novillo et. al \citep{NOVILLO2024114355}. Combining
different elements of the events dataset, we can create a graph
corresponding to the passing network of each team, allowing us to
understand the passing structure of both teams.

Figs. \protect\hyperlink{f2.3}{2.3} and
\protect\hyperlink{f2.4}{2.4} illustrate the
passing networks observed in the Eibar versus Málaga football match,
providing insight into the passing interactions and tactical strategies
used by both teams. The nodes in the graphs represent individual players
who participated in the match for each team. The nodes are sized
according to their degree, which represents the amount of ingoing and
outgoing passes. The node position corresponds to the average passing
position of each player. Substitutes are represented by yellow nodes,
and links are created if there have been at least 5 passes made in that
direction between two players. The edge's width corresponds to the
amount of passes made in that direction between the two players.

Representation of the Eibar passing networks of the match
Eibar - Málaga. Nodes represent players, edges represent passes between
players. The position of the players in the field is their average
passing position. The size of the nodes reflects the number of ingoing
and outgoing passes (i.e.~node's degree), while the size of the edges is
proportional to the number of passes between the players. Substitutes
are represented in yellow. A connection is set if those players share at
least 5 passes. The edge's width is proportional to the amount of passes
made in that direction between the two players.

Representation of the Málaga passing networks of the match
Eibar - Málaga. Nodes represent players, edges represent passes between
players. The position of the players in the field is their average
passing position. The size of the nodes reflects the number of ingoing
and outgoing passes (i.e.~node's degree), while the size of the edges is
proportional to the number of passes between the players. Substitutes
are represented in yellow. A connection is set if those players share at
least 5 passes. The edge's width is proportional to the amount of passes
made in that direction between the two players.

Analysis as the former can be conducted \emph{in real-time}\footnote{Opta uses a combination of human annotation, computer vision, and
  AI modelling to offer real-time data at various levels of detail
  based on customer requirements. In our situation, the data feed
  updates itself when an event such as a goal, foul or pass occurs;
  otherwise, it updates every 90 seconds. \citep{opta}} during the
match using appropriate data sources. Additionally, we could examine
Eibar's macro situation during the 2022-2023 season to better comprehend
how this micro-statistics contribute to the overall perception of the
team.

Fig. \protect\hyperlink{f2.5}{2.5} presents the
expected goals (xG) produced by Eibar in all matches played against
their opponents. It is noticeable that Eibar has generated a higher xG
when playing at their Home stadium, on average. In Fig.
\protect\hyperlink{f2.6}{2.6} an overview of
Eibar's performance against other teams in the Second Division is
presented. It can be observed that Eibar ranks third in generating xGs
against their opponents.

Expected Goals (xG) and Expected Goals Against (xGA) per
match. Codes: Home Matches (Diamonds), Away Matches (Circles), Wins
(Green), Draws (Blue), Losses (Red). Matches above the dashed lines
represent those matches where Eibar has generated more xG than the
opponent.

Ranking of the average differences in Expected Goals Scored
(xG) Minus Expected Goals Conceded by Opponents (xGa) per
team.

We've just discussed some of the many statistics that can be inferred
from this data sources to characterize the team performance, such as
possession, pressure, duels, fouls, etc. Due to limitations in space and
scope, however, we are unable to provide a more in-depth analysis of
these measures\footnote{For additional information regarding this type of analysis, we
  have included two reports outlining Eibar's performance during the
  2022/2023 season in the annex. In these reports, an array of metrics
  is presented to appraise team performance relative to other teams in
  the league, as well as each individual player's performance.}.

Although event datasets supply beneficial information regarding the
team's overall performance, deeper scrutiny can be conducted via
tracking data, which consists of the players' and ball's position and
movement during the match. Tracking data can offer additional insight
into both the physical and tactical aspects of the game.

\hypertarget{tracking-1}{%
\subsection{Tracking}\label{tracking-1}}

Tracking data offers a more comprehensive perspective than event data by
providing access to information on all players, their trajectories, and
velocities. This allows for the analysis of off-ball players and team
dynamics, resulting in a more nuanced understanding of the game.

There are two main techniques for obtaining tracking data, which decide
its classification: Image detection algorithms extract players'
positions from the match broadcast and infer locations of concealed
players, whereas optical tracking employs a specialized camera system
installed on the field to record players' data. Our research will
concentrate on the latter method, as it offers more precise and
statistically informative data.

Mediacoach® utilises the Tracab Optical Tracking system to obtain
on-the-pitch player positions. This multi-camera system captures each
player's position at 25 frames per second. The system consists of three
units, each with a resolution of 1920x1080 pixels, producing a panoramic
picture that generates a stereoscopic view for triangulating the players
and ball. In case of a temporary loss of any location, a skilled
operator adjusts the players' positions. The datasets obtained by the
Mediacoach® system have been validated in advance using GPS
\citep{Felipe2019ValidationOA}.

As a demonstration of the amount of information this kind of data can
provide us, we have painted a frame from Atlético de Madrid (Blue) -
Getafe (Red) game from the Spanish 2019 League. Fig.
\protect\hyperlink{f2.7}{2.7}. The ball is shown
as a black dot. Referees are shown as yellow squares. Purple arrows
represent the speed vectors of the players.

As a first approach, we can see where the players have moved during the
game by plotting the heat map of the ball over the whole game. Note that
we always keep the direction of play from left to right, so the home
team will always be placed on the left side of the field and the away
team on the right. Fig. \protect\hyperlink{f2.8}{2.8}.

A frame of tracking data from a football match. The home
team is shown in blue, the away team in red. The ball is shown as a
black dot. Referees are shown as yellow squares. Purple arrows represent
the speed vectors of the players.

Heatmap of the ball position during the Atlético de Madrid -
Getafe game under study. Note that we always keep the direction of play
from left to right, so the home team will always be placed on the left
side of the field and the away team on the right.

Heatmaps such as the one shown in Fig. \protect\hyperlink{f2.8}{2.8} have proved useful to a wide range of stakeholders,
including technical staff, journalists and passionate fans, who have all
sought to gain insight into the positioning of players during matches.
Such granular information about a player's spatial occupation of the
pitch serves as a valuable resource for technical staff, both in match
preparation by analysing the expected positions of opponents, and in
post-match evaluation to assess the extent to which the tactical game
plan has been adhered to. \citep{Garrido_2022}.

This data-driven approach enhances the understanding of the sport and
its strategic nuances, fostering a deeper appreciation of the game's
intricacies. For this reason, tracking data has gained popularity as a
valuable tool in football analysis, providing insights into player
performance and team strategy. These advanced metrics can be applied to
specific games, as in the analysis presented earlier (Fig.
\protect\hyperlink{f2.8}{2.8}), or to an ensemble
of them to provide a comprehensive view of general player behavior under
different parameters. In the following sections, we present a selection
of these metrics, both physical and tactical, to provide context for our
proposed Offside Control metric across multiple games.

\hypertarget{physical-metrics-1}{%
\subsection{Physical metrics}\label{physical-metrics-1}}

The analysis of player speed in football is a topic of great interest in
sports science and performance analysis. Understanding the role and
development of sprinting speed in football is crucial to optimizing
player performance. Using tracking data, we can make general statistics
about the patterns that are evident in the speed of football players. In
Fig. \protect\hyperlink{f2.9}{2.9} we analyzed 362
players in 100 matches with a total of 1,502,560 frames. To ensure the
interpretability of the results, we only included player data for those
matches in which that particular player played the entire match. The
resulting speed data showed a bimodal distribution that can be
identified with the two main actions performed by the players during the
game: walking and jogging.

Probability distribution function of the velocities for
(A) all different field players and
(B) referees during a match. Values are averaged over
{100} matches

Having characterized the general distributions underlying on players'
movement, we can delve into more particular investigations that offer
significant viewpoints on game strategy and player effectiveness.

One such question may be to check if players' performance is consistent
across the entire game. That is, if there are no significant differences
between the games' first and second halves.

In the hypothesis contrast carried out in Fig.
\protect\hyperlink{f2.10}{2.10}, significant
variations are evident between the average speeds during the yellow
regime (walking) and green regime (jogging) for all positions, between
the first and second halves of the game. This leads to the conclusion
that the average speed is lower in the latter phase. It is noteworthy
that there was a higher occurrence of values within the walking regime
in the second half compared to the first, as evidenced by the
distribution difference.

We can also observe the same outcome by analysing the mean distance
covered by each player in each half, as measured in
\protect\hyperlink{f2.11}{2.11}. The graphs
illustrate that most players remain below the diagonal dashed black
line, suggesting that the distance covered during the first part of the
game is generally grater than the one covered in the second half

Speed distribution difference between the first and second
half of the game. (A) Goalkeeper. (B)
Defender (C) Midfielder. (D) Forward.
Yellow-shadowed areas represent the values taken for the first peak.
Green-shadowed areas for the second. For each position and peak, the
average speed is smaller in the second half, with a {pvaluep \textless{} 0.001}.
* Average speed is greater in the first
half.

Distance covered in the first and second half of the game
by the players (while walking). (A) Goalkeeper.
(B) Defender (C) Midfielder.
(D) Forward. The percentages shown account for the
number of points that lie below/above the diagonal line

\hypertarget{tactical-metrics-1}{%
\subsection{Tactical metrics}\label{tactical-metrics-1}}

After analysing the physics of player movement, a sophisticated
framework is needed to decode how the intricate movements of players
translate to the soccer field.

These models provide a scientific perspective for analysing player
positioning, decision-making, and team dynamics, illuminating the
complex interactions that occur during a match.

\textbf{Pitch control} models \citep{Spearman} in soccer are an essential tool for
researchers, coaches, and analysts who aim to understand the game's
nuances at a granular level.

\hypertarget{pitch-control-1}{%
\subsubsection{Pitch Control}\label{pitch-control-1}}

The Pitch Control (PC) at a given location represents the probability of
a player or team gaining control of the ball if it moves directly to
that location. PC models simulate the dynamics of the ball and the
players to evaluate which player would control the ball if it moves to
any location on the pitch at any moment. The model captures not only the
players' current positionbut also their movement. When players are
running at high speeds, they are more likely to control the space they
are moving into rather than the space they currently occupy.

To construct this model, we must calculate the following for a given
location on the pitch:

\begin{itemize}
\item
  How long it would take for the ball to reach to the position of
  interest (from its starting position).
\item
  How long would it take for each player to get to that position.
\item
  What is the total \emph{probability} that each team will control the ball
  \emph{after} both the players and the ball have arrived at the desired
  position?
\end{itemize}

The ball is set to move at a constant speed of \(v_b = 54\) \(km/h\).
Therefore, the time taken to arrive at the location of interest can be
easily calculated as \(t_{b,arr} = \Delta x_b/v_b\), where \(\Delta x_b\) is
the distance between the initial and final positions of the ball.

When considering how long it will take the players to reach the target
position, given their initial position and speed, players are assumed to
only have a maximum speed of \(v_{max,p} = 18\) \(km/h\). This upper limit
should not be misunderstood as the maximum speed at which players can
move, but rather as an estimate of the maximum speed at which they are
likely to move when trying to control the ball.

To compute the player's expected arrival time,
\(\tau_{exp}(\vec{r} ; t_r)\), we use a simple approximation consisting of
a two-step process:

\begin{itemize}
\item
  There is an initial \emph{reaction time} of \(t_{r} = 0.7\) seconds. This
  is approximately the time it takes a player moving at maximum speed
  to come to a complete stop. During this reaction time, we assume
  that players continue to move along their current trajectory without
  changing speed or direction (reaching a position \(\vec{r}_{react}\)).
\item
  After this time, we assume that the player runs directly towards the
  ball at his maximum speed of \(v_{max,p}\).
\end{itemize}

\[\tau_{exp}(\vec{r} ; t_r) = t_r + \frac{|\vec{r} - \vec{r}_{react}|}{v_{max,p}}
    \label{exp_arr_time}\]

Once we computed the time it takes for the ball and the players to get
to the target location, we need to look at how long it will take each
player to control the ball. To do so, we will assume that controlling
the ball is a stochastic process that follows an exponential
distribution with a fixed rate \(\lambda\), with units of \(1/s\). Thus, for
any differential time \(\Delta t\) that a player is near the ball, he has
a probability of \(\lambda \cdot \Delta t\) of controlling the ball.

Cumulative distribution function of the time to control the
ball. The parameters choice is based on {}

So far, the model assumes that we know exactly when each player will
arrive at the target location. However, we introduce some uncertainty,
labelled \(\sigma\), in the arrival time of the players. The reason for
including such temporal variability in our model is to account for some
effects that have not been explicitly modelled, such as player effort.
Thus, the probability of a player intercepting the ball at time T is
given by the cumulative distribution function of the sigmoid
distribution:

\[F_{\text {int }}(\vec{r},T;\sigma, t_r)=\frac{1}{1+e^{-\frac{T- \tau_{exp}(\vec{r} ; t_r)}{\sqrt{3} \sigma / \pi}}}
    \label{sigmoid}\]

Cumulative distribution function of the probability to
intercept the ball at a certain time T {[}sigmoid{]}. {}.

Finally, the differential equation used to compute the control
probability for each player at a given location \(r\), at time t is
\citep{Spearman}:
\[\frac{d P P C F_j}{d t}\left(T, \vec{r} , \sigma, \lambda_j, t_r\right)=\left(1-\sum_k P P C F_k\left(t, \vec{r} , \sigma, \lambda_k\right)\right) F_{int,j}(t, \vec{r}  , \sigma, t_r) \lambda_j
\label{PC_eq}\] where \(PPCF_j\) is the Potential Pitch Control Field of
player \(j\). \(F_{int,j}(t, \vec{r}, T ; \sigma, t_r)\) is the probability
that player \(j\) can reach the target location \(r\) in a given time \(t\),
and \(\lambda_j\) is the control rate of such a player. We assign the
goalkeepers to have a higher control rate, \(\lambda_{GK} = 12.9\)
\(s^{-1}\), to ensure that they are likely to claim the ball if it is near
them and also to account for the ability of grabbing the ball with their
hands. Importantly, note that
\(\sum_k P P C F_k\left(T, \vec{r} , \sigma, \lambda_k\right)\) accounts
for the sum of the Potential Pitch Control Field of the rest of the \(k\)
players on the pitch at time \(t\).

By integrating Eq. \protect\hyperlink{PC_eq}{\[PC_eq\]} over \(t \in \left[ t_{ball},t_{ball} + 10 \right]\)s
and taking \(P P C F_j\left(t, \vec{r} , \sigma, \lambda_j\right) = 0\) at
the beginning of the integration, the probability of control per player
is obtained. This probability is then extracted along all the pitch,
obtaining a pitch control surface, which can be plot and interpreted by
scientists, technical staff or players.

\hypertarget{offside-control-1}{%
\chapter{Offside Control}\label{offside-control-1}}

This study introduces the Offside Control (OC) parameter, which is
derived from a modification of the Potential Pitch Control Field to
analyse the pitch control generated by teams/players beyond the offside
line. The offside line is identified as the location of the
second-to-last defender, and the attacking players in both valid and
offside positions are detected at every frame. The OC is calculated as
the pitch control generated by the attacking team/player after the
offside line. Additionally, if a player is in an offside position, their
contribution is quantified as Ineffective Offside Control (IOC).
Conversely, when a player is before the offside line and generates pitch
control after it, it is referred to as Effective Offside Control (EOC).

Fig. \protect\hyperlink{f3.1}{3.1} illustrates the
concept of effective and ineffective Offside Control. The players of
both teams are represented in cyan (attacking from left to right) and
red (attacking from right to left), corresponding to the home and away
team, respectively. The regions of the pitch controlled by the home team
are indicated in blue, while the red regions are controlled by the away
team. The offside position is indicated by the vertical dashed line. The
wide grey line surrounds the areas controlled by the attacking team at
the back of the offside line. Player \(11\), is located behind the offside
line, which is an invalid position and generates an offside infraction.
In contrast, player \(18\) is in a correct position, and the offside
control generated behind the offside line is classified as effective
(EOC).

In the study, we computed the Offside Control (both effective and
ineffective) at a rate of 2 frames per second for each of the 100
matches analyzed, resulting in a total of 1,251,934 frames analysed.
Doing so, we managed to characterize the Offside Control of 442 players
in total. To efficiently discretize the space, the computations of the
Offside Control were confined to the attacking half of the pitch of the
team in possession of the ball. This method considerably decreased the
model's computation time while maintaining a high spatial resolution of
50 by 32 divisions of the attacking half of the pitch.

Example of effective (EOC) and ineffective (IOC) Offside
Control (areas surrounded by a wide grey line). The offside line is the
vertical dashed line. Note that player {11} is offside (generating IOC), while player
{18} is not (creating EOC).

To obtain the OC generated by the team, simply sum the individual
contributions of the players during the match. Figure 3.2 shows two
examples of the different spatial distribution of the EOC in two
different matches of the same team (Team 1). In MATCH A, we can see how
the exerted EOC by Team 1 is deep and close to the opponent's area (Team
2), whereas in the second match, MATCH B, the EOC is longitudinal across
the side lanes. The bar chart on the right-hand side of each plot
displays the average EOC generated by each team per unit of effective
time, considering only instances where the ball was in play. In both
cases, Team 1 created more pressure after the offside lane than their
opponents.

Cumulative Effective Offside Control distribution over the
course of two matches. The same team (Team 1) (left) generates offside
control in two different ways at two different matches (A,B). Bars on
right-hand side show the average effective offside control generated by
each team per unit of effective time. {ºbf (cambiar numeros por
letras y viceversa, segun texto principal)}

The correlation between the average OC per unit of effective time and
the final classification following the tenth match of the season is
shown in Fig. \protect\hyperlink{f3.3}{3.3}. The
scatter plot shows both the generated and received average OC per unit
of effective time, with each label and point size representing the
team's classification after the tenth match of the season. Teams above
the diagonal black dashed line have generated more OC than their
opponents.

Average generated Offside Control vs the received one. The
values correspond to the average accumulated OC per effective unit of
time. The labels and the size of each point correspond to their final
classification after the tenth match of the season. Teams above the dark
dashed line have accumulated more offside control than the one generated
by their rivals. The fitted equation is {y =  − 0.86x + 285.42}, with
an {R2 = 0.48} and
{RMSE = 14.96}

The distribution of pressure among teams in the league is heterogeneous,
with some higher-ranking teams exerting more pressure than their
opponents. There are some exceptions, with lower-ranking teams also
exerting more pressure than their opponents. However, it is clear that
OC alone cannot fully explain team rankings. A more comprehensive study
is needed to include additional variables that could better explain the
rankings.

Fig. \protect\hyperlink{f3.4}{3.4} illustrates the
progression of Cumulative Offside Control during a match. The cumulative
sum is calculated using a centered moving window comprising 600 frames.
The blue line represents the total offside control of the home team,
including both effective and ineffective contributions, per unit of
effective time.

The cumulative rate of the OC decreases towards the latter half of the
game, resulting in the home team losing pressure on the opponent and
conceding a couple of goals.

Cumulative Offside Control evolution over the course of a
match. Cumulative evolution of offside control over the course of a
match. The cumulative sum has been taken using a centered moving window
of 600 frames. the total offside control of the home team is represented
in blue (including both effective/ineffective contributions) per unit of
effective time.

By examining the spatial distribution of the total OC generated across
all matches played and studying the forwards' positions relative to the
offside line, we find that OC is contingent on the forwards' position
relative to the offside line, thus proving useful in characterising
different forward playstyles. Figure \protect\hyperlink{f3.5}{3.5} shows that Forward A spends more time preceding the
offside line (negative values) than Forward B. The EOC is normalised by
the time played and encompasses a minimum of 6 matches.

Effective Offside Control of two different forwards A/B. On
the left, positions of the pitch where two different forward players A/B
generate Effective Offside Control (EOC). On the right, a probability
distribution function of the distance to the offside line for the two
different forwards. EOC is normalized by the time played, containing at
least 6 matches. Note how the spatial distribution of the EOC depends on
their position with respect to the offside line. We can see of forward A
spends more time before the offside line (negative values) than forward
B.

\hypertarget{conclusion-1}{%
\chapter{Conclusion}\label{conclusion-1}}

Data-driven approaches for analysing football performance using event
and tracking data have gained popularity in the last decade. In this
project, we previously discussed how metrics such as Expected Goals
(xG), derived from events datasets, have impacted our understanding of
the game, becoming an essential tool for evaluating team performance.

However, tracking datasets allow us to develop advanced physical and
tactical metrics that capture the aspects of player and team
performance. In this context, Pitch Control arises as sophisticated
approach to quantifying and analyzing the intricate dynamics of soccer
from a scientific perspective. Pitch control models aim to provide a
quantitative framework for assessing player movements, spatial control,
and decision-making on the field. They utilize positional data of
players and the ball to determine the likelihood of a player gaining
control. At any given moment, the position of the ball can be determined
by taking into account variables such as player speed, direction, and
relative positioning to their opponents and teammates. This modelling
approach provides tactical insights and spatial relationships between
players, the ball, and the field. By studying how players' movements
affect pitch control, teams can optimize their formations, pressing
patterns, and passing strategies.

From these models, metrics such as the one presented in this project,
Offside Control, quantify a player's pitch control contributions locally
(in this case, after the offside line). This makes it possible to
identify strengths and weaknesses in team pressure, track improvements
over time, and make data-driven decisions regarding player selection and
development. Clubs and talent scouts can use these metrics derived from
Pitch Control models to evaluate potential signings. In the case of
Offside Control, these models can aid in identifying talented players by
quantifying their ability to control space and make effective decisions.

Currently, tracking dataset approaches are not measured in real-time,
unlike events dataset. The development of new technologies for football
analytics, such as the recently announced TRACAB Optical system, shows
promise for research in this area \citep{TRACAB}. The system uses advanced AI
to provide live in-game tracking data and unprecedented levels of
precision to identify and track players' body parts. This technology can
power new categories of performance analysis and skeletal modeling for
exciting new applications. As the industry evolves, so will the
sophistication of these models, offering a deeper understanding of the
game and enhancing its scientific appeal.

\bibliography{bibliografia.bib}



\end{document}
