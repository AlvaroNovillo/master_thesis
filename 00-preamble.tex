\geometry{
  a4paper,
  top=2.5cm,
  bottom=2.5cm,
  inner=3cm,
  outer=3cm
}


\renewcommand{\baselinestretch}{1.15}
\parskip=6pt

\definecolor{azulUC3M}{RGB}{0,0,102}
\definecolor{gray97}{gray}{.97}
\definecolor{gray75}{gray}{.75}
\definecolor{gray45}{gray}{.45}

\usepackage[a-1b]{pdfx}

\usepackage{hyperref}
\hypersetup{colorlinks=true,
	linkcolor=black, % enlaces a partes del documento (p.e. índice) en color negro
	urlcolor=blue} % enlaces a recursos fuera del documento en azul


\usepackage{amsmath,amssymb,amsfonts,amsthm}

\usepackage{txfonts} 
\usepackage[T1]{fontenc}
\usepackage[utf8]{inputenc}

\usepackage[english]{babel} 
\usepackage[babel, english=american]{csquotes}
\AtBeginEnvironment{quote}{\small}

% diseño de PIE DE PÁGINA
\usepackage{fancyhdr}
\pagestyle{fancy}
\fancyhf{}
\renewcommand{\headrulewidth}{0pt}
\rfoot{\thepage}
\fancypagestyle{plain}{\pagestyle{fancy}}


\usepackage{titlesec}
\usepackage{titletoc}
\titleformat{\chapter}[block]
{\large\bfseries\filcenter}
{\thechapter.}
{5pt}
{\MakeUppercase}
{}
\titlespacing{\chapter}{0pt}{0pt}{*3}
\titlecontents{chapter}
[0pt]                                               
{}
{\contentsmargin{0pt}\thecontentslabel.\enspace\uppercase}
{\contentsmargin{0pt}\uppercase}                        
{\titlerule*[.7pc]{.}\contentspage}                 

\titleformat{\section}
{\bfseries}
{\thesection.}
{5pt}
{}
\titlecontents{section}
[5pt]                                               
{}
{\contentsmargin{0pt}\thecontentslabel.\enspace}
{\contentsmargin{0pt}}
{\titlerule*[.7pc]{.}\contentspage}

\titleformat{\subsection}
{\normalsize\bfseries}
{\thesubsection.}
{5pt}
{}
\titlecontents{subsection}
[10pt]                                               
{}
{\contentsmargin{0pt}                          
	\thecontentslabel.\enspace}
{\contentsmargin{0pt}}                        
{\titlerule*[.7pc]{.}\contentspage}  



\usepackage{multirow} %permite combinar celdas 
\usepackage{caption} %para personalizar el título de tablas y figuras
\usepackage{floatrow} %utilizamos este paquete y sus macros \ttabbox y \ffigbox para alinear los nombres de tablas y figuras de acuerdo con el estilo definido. Para su uso ver archivo de ejemplo 
\usepackage{array} % con este paquete podemos definir en la siguiente línea un nuevo tipo de columna para tablas: ancho personalizado y contenido centrado
\newcolumntype{P}[1]{>{\centering\arraybackslash}p{#1}}
\DeclareCaptionFormat{upper}{#1#2\uppercase{#3}\par}


% Diseño de tabla para ingeniería
\captionsetup[table]{
	format=upper,
	justification=centering,
	labelsep=period,
	width=.75\linewidth,
	labelfont=small,
	font=small,
}




\usepackage{graphicx}
\graphicspath{{imagenes/}} %ruta a la carpeta de imágenes

% Diseño de figuras para ingeniería
\captionsetup[figure]{
	format=hang,
	name=Fig.,
	singlelinecheck=off,
	labelsep=period,
	labelfont=small,
	font=small		
}




\usepackage{chngcntr}
\counterwithout{footnote}{chapter}


\usepackage{listings}


\lstdefinestyle{estilo}{ frame=Ltb,
	framerule=0pt,
	aboveskip=0.5cm,
	framextopmargin=3pt,
	framexbottommargin=3pt,
	framexleftmargin=0.4cm,
	framesep=0pt,
	rulesep=.4pt,
	backgroundcolor=\color{gray97},
	rulesepcolor=\color{black},
	%
	basicstyle=\ttfamily\footnotesize,
	keywordstyle=\bfseries,
	stringstyle=\ttfamily,
	showstringspaces = false,
	commentstyle=\color{gray45},     
	%
	numbers=left,
	numbersep=15pt,
	numberstyle=\tiny,
	numberfirstline = false,
	breaklines=true,
	xleftmargin=\parindent
}

\captionsetup[lstlisting]{font=small, labelsep=period}

\lstset{style=estilo}
\renewcommand{\lstlistingname}{\uppercase{Código}}
